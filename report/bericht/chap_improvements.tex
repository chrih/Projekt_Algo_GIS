\section{Optimierungsansätze}
Die implementierten Algorithmen lassen weitere Optimierungen zu.
% - Priortätswarteschlange klein halten
Ein Optimierungsansatz, der bereits implementiert wurde, ist, die Prioritätswarteschlange der zukünfigen Ereignisse klein zu halten.
Dies wird erreicht, indem die Ereignisobjekte nach und nach erstellt und in der Warteschlange eingereiht werden,
anstatt dies gleich am Anfang des Algorithmus für alle Pixel zu tun.
Details dazu sind in Abschnitt~\ref{ev_klein} zu finden.

% - parallelisieren des Problems (Aufteilen des grids in Teile und einzeln prozessieren)
Weiterhin kann die Berechnung des Viewsheds parallelisiert werden.
Dies ist für den naiven Algorithmus besonders leicht, weil die Sichtbarkeit jedes Pixel unabhängig von allen anderen berechnet wird.
Da der naive Algorithmus mit Java-\emph{Streams} implementiert wurde und diese Parallelisierung sprachseitig unterstützen,
konnte der Algorithmus ohne Aufwand parallelisiert werden.
Bei dem Sweep-Line-Algorithmus müssen die Pixel anders aufgeteilt werden, um die Vorteile der Sweep-Line weiterhin ausnutzen zu können.
Man teilt hier die Ebene an bestimmten Winkeln auf, zum Beispiel in die vier Quadranten (das heißt an den Winkeln $0^\circ$, $90^\circ$, $180^\circ$ und $270^\circ$).
Dann überstreichen mehrere Sweep-Lines gleichzeitig die Ebene und beschleunigen so die Berechnung des Viewsheds.

% - Quadtrees
Ein weiterer Optimierungsansatz ist, den Viewshed erst nur grob zu berechnen -- also für aggregierte Pixel -- und dann graduell zu verfeinern.
Man teilt also die Ebene hierarchisch in Blöcke auf und speichert für jeden Pixelblock die minimale und maximale Höhe.
Wenn ein Pixelblock auf seiner Maximalhöhe von einem Block mit seiner Minimalhöhe verdeckt wird, kann der komplette Block als nicht sichtbar markiert werden.
Analog kann ein Pixelblock auf seiner Minimalhöhe, der von keinem Block auf seiner Maximalhöhe verdeckt wird, als sichtbar markiert werden.
Man vermeidet unter Umständen also die Berechnung der Sichtbarkeit einiger Pixel, beispielsweise wenn eine Erhebung die ganze Landschaft hinter sich verdeckt.
