\section{Einleitung}
In unserem Projekt zur Vorlesung ``Algorithmen für geografische Informationssysteme'' stellen wir das Problem der ``Viewshed''-Analyse vor und 
vergleichen verschiedene Lösungs-Algorithmen. \\ Der englische Begriff \textit{viewshed} ist definiert als \textit{die geografische Fläche, die 
von einem bestimmtem Standpunkt aus sichtbar ist}. Der viewshed wird beispielweise bei folgenden Problemstellungen benötigt:
\begin{itemize}
 \item Bestimmung günstiger Standorte für Sendemasten
 \item Finden von Aussichtspunkten (z.B. im Gebirge), um Wanderwege planen zu können
 \item Finden von versteckten Routen (z.B. für militärische Zwecke)
 \item Finden von besonders schönen Routen, z.B. entlang der Küste 
\end{itemize}
Um den viewshed bestimmen zu können, ist ein sogenanntes \textit{digital elevation model} (DEM) nötig. 
\noindent Ein DEM ist eine zwei- oder dreidimensionale Darstellung einer geografischen Karte und enthält die Höhendaten des dargestellten Gebiets. 
Für unsere Problemstellung ist die zweidimensionale Variante ausreichend.

\noindent In einem DEM, welches als zweidimensionales Array dargestellt werden kann, wird ein Punkt als Standort gewählt. Um z.B. die Höhe eines 
Sendemasten miteinzuberechnen, kann auch eine zusätzliche Höhe angegeben werden, welche auf die Höhe des Standorts addiert wird.

\noindent Als Testdaten wurden DEMs der Bayerischen Vermessungsverwaltung \cite{berchtesgaden} und des Salzburger Geographischen Informationssystems 
\cite{salzburg} verwendet. Da die Testdaten in einer simplen Textdatei (siehe Abbildung \ref{testfile}) zur Verfügung gestellt werden, lassen sich die Daten leicht 
auslesen und weiterverarbeiten.

\begin{figure}[!ht]
 \centering
 \begin{BVerbatim}
ncols         501
nrows         1001
xllcorner     4490660
yllcorner     5320200
cellsize      2
558.21 558.13 558.08 557.99 557.93 557.81 557.7  ....
558.1 558.06 557.95 557.89 557.83 557.73 557.6 ....
557.91 557.85 557.78 557.75 557.67 557.58 557.47 ...
557.81 557.78 557.7 557.66 557.56 557.48 557.4  ...
557.64 557.6 557.54 557.46 557.44 557.36 557.23  ...
557.59 557.51 557.46 557.38 557.37 557.28 557.15 ...
...
\end{BVerbatim}
\caption{DEM-Datei}
\label{testfile}
\end{figure}

\noindent Als Programmiersprache wurde Java (Version 1.8) verwendet. 

