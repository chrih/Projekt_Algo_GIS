\section{Einleitung}
In unserem Projekt zur Vorlesung ``Algorithmen für geografische Informationssysteme'' stellen wir das Problem der ``Viewshed''-Analyse vor und 
vergleichen verschiedene Lösungs-Algorithmen. \\ Der englische Begriff \textit{viewshed} ist definiert als \textit{die geografische Fläche, die 
von einem bestimmtem Standpunkt aus sichtbar ist}. Der viewshed wird beispielweise bei folgenden Problemstellungen benötigt:
\begin{itemize}
 \item Bestimmung günstiger Standorte für Sendemasten, damit möglichst eine große Fläche abgedeckt wird
 \item Finden von Aussichtspunkten (z.B. im Gebirge), um Wanderwege planen zu können
 \item Finden von versteckten Routen (z.B. für militärische Zwecke)
 \item Finden von besonders schönen Routen, z.B. entlang der Küste 
\end{itemize}
Um den viewshed bestimmen zu können, ist ein sogenanntes \textit{digital elevation model} (DEM) nötig. Ein DEM ist eine zwei- oder 
dreidimensionale Darstellung einer geografischen Karte und enthält die Höhendaten des dargestellten Gebiets. Für unser Problem ist die 
zweidimensionale Variante ausreichend. In diesem DEM, welches in unserem Programm als zweidimensionales Array dargestellt wird, wird ein Punkt als 
Standort gewählt. Um z.B. die Höhe eines Sendemasten miteinzuberechnen, kann auch eine zusätzliche Höhe angegeben werden, welche auf die Höhe des 
Standorts addiert wird.

\cite{Fisher} \cite{vanKrev}

