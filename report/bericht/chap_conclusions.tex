\section{Fazit}
In diesem Projekt untersuchten wir M�glichkeiten, den Viewshed eines Punktes zu berechnen.
Als Basis wurde ein naiver Algorithmus mit einer Laufzeit von $O(n^3)$ bei einer Seitenl�nge von $n$ implementiert.
Weiterhin wurde ein effizienter Algorithmus implementiert, der von van Kreveld~\cite{vanKrev} vorgestellt wurde und eine Laufzeit von $O(n^2\log(n))$ ben�tigt.
Die Algorithmen wurden auf unterschiedliche Weise optimiert, wie durch Parallelisierung oder Verwendung alternativer Datenstrukturen.
Die Implementierung dieser verschiedenen Algorithmen zur Berechnung des Viewsheds hat gezeigt, dass die Laufzeit stark von der Eingabe abh�ngt.
Je kleiner die Pixelgr��e des DEMs, desto h�her die Laufzeit -- desto genauer aber auch die Ergebnisse.
Wie jedoch in Abbildung \ref{dgm_erg} gezeigt, ist bei einer Pixelgr��e von 1 bis 10 Metern kaum ein Unterschied mit dem blo�en Auge zu erkennen.
Somit l�sst sich einiges an Rechenzeit einsparen,
wenn man zun�chst einen groben Viewshed mit einer gr��eren Pixelgr��e (und somit kleinere Aufl�sung) berechnet
und dann -- falls gew�scht -- kleine Teile des DEMs genauer betrachtet.
Weitere m�gliche Ans�tze zur Verbesserung der Laufzeit wurden in Kapitel~\ref{opt_an} vorgestellt.
