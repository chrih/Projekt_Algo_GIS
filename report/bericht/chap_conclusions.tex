\section{Fazit}
In diesem Projekt untersuchten wir Möglichkeiten, den Viewshed eines Punktes zu berechnen.
Als Basis wurde ein naiver Algorithmus mit einer Laufzeit von $O(n^3)$ bei einer Seitenlänge von $n$ implementiert.
Weiterhin wurde ein effizienter Algorithmus implementiert, der von van Kreveld~\cite{vanKrev} vorgestellt wurde und eine Laufzeit von $O(n^2\log(n))$ benötigt.
Die Algorithmen wurden auf unterschiedliche Weise optimiert, wie durch Parallelisierung oder Verwendung alternativer Datenstrukturen.
Die Implementierung dieser verschiedenen Algorithmen zur Berechnung des Viewsheds hat gezeigt, dass die Laufzeit stark von der Eingabe abhängt.
Je kleiner die Pixelgröße des DEMs, desto höher die Laufzeit -- desto genauer aber auch die Ergebnisse.
Wie jedoch in Abbildung \ref{dgm_erg} gezeigt, ist bei einer Pixelgröße von 1 bis 10 Metern kaum ein Unterschied mit dem bloßen Auge zu erkennen.
Somit lässt sich einiges an Rechenzeit einsparen,
wenn man zunächst einen groben Viewshed mit einer größeren Pixelgröße (und somit kleinere Auflösung) berechnet
und dann -- falls gewüscht -- kleine Teile des DEMs genauer betrachtet.
Weitere mögliche Ansätze zur Verbesserung der Laufzeit wurden in Kapitel~\ref{opt_an} vorgestellt.
